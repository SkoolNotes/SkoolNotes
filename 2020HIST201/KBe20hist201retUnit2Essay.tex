% Created 2020-10-31 Sat 13:26
% Intended LaTeX compiler: pdflatex
\documentclass[11pt]{article}
\usepackage[utf8]{inputenc}
\usepackage[T1]{fontenc}
\usepackage{graphicx}
\usepackage{grffile}
\usepackage{longtable}
\usepackage{wrapfig}
\usepackage{rotating}
\usepackage[normalem]{ulem}
\usepackage{amsmath}
\usepackage{textcomp}
\usepackage{amssymb}
\usepackage{capt-of}
\usepackage{hyperref}
\author{Exr0n}
\date{\today}
\title{E Unit 2 Essay Planning + Outline}
\hypersetup{
 pdfauthor={Exr0n},
 pdftitle={E Unit 2 Essay Planning + Outline},
 pdfkeywords={},
 pdfsubject={},
 pdfcreator={Emacs 27.1 (Org mode 9.3)},
 pdflang={English}}
\begin{document}

\maketitle
\tableofcontents

\section{Prompt}
\label{sec:org099cf4a}
\begin{quote}
Option 1: "The concept of the balance of power was simply an extension of conventional wisdom. Its primary goal was to prevent domination by one state and to preserve the international order; it was not designed to prevent conflicts, but to limit them. To the hard-headed statesmen of the eighteenth century, the elimination of conflict (or of ambition or of greed) was utopian; the solution was to harness or counterpoise the inherent flaws of human nature to produce the best possible long-term outcome."

Henry Kissinger, Diplomacy

From one point of view, balance of power politics in the early modern period succeeded spectacularly, preventing a single European power from conquering the whole continent, although Napoleon almost succeeded. From another point of view, it exported great power conflict to the rest of the world, turning Africa, the Middle East, Asia, and the Americas into battlegrounds for rivalling European states. In the end, did the European balance of power succeed in its goal to, as Kissinger puts it, limit conflict and produce the “best possible outcome” from flawed human nature? Or did it magnify conflict and increase the likelihood of global war? Answer this question in a well organized essay using examples from multiple global regions.  (Kissinger, Mason, Roberts)
\end{quote}
\section{Outline}
\label{sec:org5fee3fe}
\subsection{Intro}
\label{sec:org76c00f3}
\subsubsection{Thesis}
\label{sec:orga00b948}
Although Europe managed to remain disjoint and disunited, "balance of power politics" hardly limited conflict or produced a desireable <WC> outcome: Raison d'etat prolonged wars in Europe and exported conflict and oppression to India and the Caribbean.


\subsection{Richelieu + Raison d'etat}
\label{sec:orgf87fea5}
Raison d'etat--the basis <WC> for 'balance of power politics'--countered unification by prolonging wars and levaraging suffering, creating unessesary conflict for a not-so-rosy <WC> continuation of power struggle <rephrase>.
\subsubsection{"In order to prolong the war and exhaust the belligrants, Richelieu subsidized the enemeies of his enemies, bribed, formented insurrections, [etc]" (Kissinger 62)}
\label{sec:orge5cafa1}
\subsubsection{"France stood on the sidelines while Germany was devastated" (Kissinger 62)}
\label{sec:org93e2817}
\subsubsection{"He seeks peace by means of war" (Quote on Kissinger 64, footnote 10)}
\label{sec:org4290e8f}
\subsubsection{"[Richelieu believed] the end justified the means" (Kissinger 64)}
\label{sec:org159346e}


\subsection{India}
\label{sec:org94c3989}
Not only did "conflict politics" <WC> prolong <WC> conflict <WC> in Europe, it also created an air of rivalary that brought other reigons, such as India, into the fray. <Expand and link to thesis?>
\sout{\textbf{*} "India had been irresistably sucked into the worldwide conflict between British and French power" (Roberts 642)}
\subsubsection{"But the immediate cause of British rule in India was the worldwide struggle of England and France, which the English and French East India Companies joined in," (Trauttmann 177)}
\label{sec:orge317e28}
\subsubsection{"These 'princely states' \ldots{} each had a British 'resident' who kept them apprised of British policy, and often interfered with the internal goverance and the sucession to the kingdom." (Trauttmann 179)}
\label{sec:orgb5b330a}
\subsubsection{"These armed forces of the merchant companies now became war-making entities that drew Indian governments and their armies into the commercial and national struggle between the British and the French." (Trauttmann 176)}
\label{sec:org31113c2}
\subsubsection{"[The mutinies of 1806 and 1857] had elemnts of feeling that religion was under attack" (Trauttmann 179)}
\label{sec:org7d36128}
\subsubsection{"The aftermath of the faild Rebellion was a complex mixture of repression ond canciliation by the British. The mutineers themselves \ldots{} were harshly and publically punished, some of them being tied to the ends of cannon and blown in half." (Trauttmann 181)}
\label{sec:org2e690dd}

\subsection{Caribbean}
\label{sec:org696781b}
Like India, colonial rivalaries brought European exploitation and conflict to the Caribbean, wiping out the native population and driving the slave trade in the process.

\subsubsection{Rivalary}
\label{sec:org49b466c}
\begin{enumerate}
\item "The spanish occupation of the larger Caribbean islands \ldots{} attracted the attention of the English, French and Dutch" (Roberts 650)
\label{sec:org15952ed}
\item "[Tobacco colonies in the new world] rapidly became of great importance to England, not only because of the customs revenue they supplied, but also because [they] provided fresh opportunities for interloping in the trade of the Spanish empire." (Roberts 651)
\label{sec:org81e6a2c}
\item "Production was for a long time held back by a shortage of labor, as the native populations of the islands succumbed to European ill-treatment and disease." (Roberts 650)
\label{sec:org1c0a59d}
\item Lots of slaves: 6k slaves in 1643 but 50k in 1660 (Roberts 651)
\label{sec:orgc0f2d81}
\item "where colonial fronteers met and policing was poor and there were great prizes to be won, [the area] became the classical, indeed, legendary hunting ground of pirates." (Roberts 652)
\label{sec:orgf787ffb}
\end{enumerate}


\subsection{Conclusion}
\label{sec:org4730c48}



\section{Editing}
\label{sec:org1f828a6}
\subsection{WC}
\label{sec:org24c356f}
\subsubsection{{\bfseries\sffamily TODO} Need more synonyms for "balance of power politics"}
\label{sec:orge23bc32}
\begin{enumerate}
\item power politics?
\label{sec:orgb736abc}
\item conflict politics?
\label{sec:org700bf65}
\end{enumerate}
\end{document}
